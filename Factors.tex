%====================================================================================================%

Factors
A factor is a special type of vector used to represent categorical
data, e.g. gender, social class, etc.
 Stored internally as a numeric vector with values 1, 2, : : : ; k,
where k is the number of levels.
 Can have either ordered and unordered factors.
 A factor with k levels is stored internally consisting of 2 items
(a) a vector of k integers
(b) a character vector containing strings describing what the k
levels are.


%===========================================================================================%
\begin{frame}

Factors
Example
Consider a survey that has data on 200 females and 300 males. If
the rst 200 values are from females and the next 300 values are
from males, one way of representing this is to create a vector
> gender <- c(rep("female", 200), rep("male", 300))
To change this into a factor
> gender <- factor(gender)
The factor gender is stored internally as
1 female
2 male
Each category, i.e. female and male, is called a level of the factor.
To determine the levels of a factor the function levels() can be
used:
> levels(gender)
[1] "female" "male"

Factors
Example
Five people are asked to rate the performance of a product on a
scale of 1-5, with 1 representing very poor performance and 5
representing very good performance. The following data were
collected.
> satisfaction <- c(1, 3, 4, 2, 2)
> fsatisfaction <- factor(satisfaction, levels=1:5)
> levels(fsatisfaction) <- c("very poor", "poor", "average",
"good", "very good")
The rst line creates a numeric vector containing the satisfaction
levels of the 5 people. Want to treat this as a categorical variable
and so the second line creates a factor. The levels=1:5 argument
indicates that there are 5 levels of the factor. Finally the last line
sets the names of the levels to the specied character strings.

%===========================================================================================%
