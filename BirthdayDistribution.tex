The Birthday function
The R command pbirthday() computes the probability of a coincidence of a number of randomly chosen people sharing a birthday, given that there are n people to choose from.
Suppose there are four people in a room. The probability of two of them sharing a birthday can be computed as follows:
> pbirthday(4)
[1] 0.01635591

Answer:  about 1.6 %
How many people do you need for a greater than 50% chance of a shared birthday?
(starting from 2 – so the 5th element corresponds to 6 people, etc)

 > sapply(2:60,pbirthday)
 [1] 0.002739726 0.008204166 0.016355912 0.027135574
 [5] 0.040462484 0.056235703 0.074335292 0.094623834
 [9] 0.116948178 0.141141378 0.167024789 0.194410275
[13] 0.223102512 0.252901320 0.283604005 0.315007665
[17] 0.346911418 0.379118526 0.411438384 0.443688335
[21] 0.475695308 0.507297234 0.538344258 0.568699704
[25] 0.598240820 0.626859282 0.654461472 0.680968537
[29] 0.706316243 0.730454634 0.753347528 0.774971854
[33] 0.795316865 0.814383239 0.832182106 0.848734008
[37] 0.864067821 0.878219664 0.891231810 0.903151611
[41] 0.914030472 0.923922856 0.932885369 0.940975899
[45] 0.948252843 0.954774403 0.960597973 0.965779609
[49] 0.970373580 0.974431993 0.978004509 0.981138113
[53] 0.983876963 0.986262289 0.988332355 0.990122459
[57] 0.991664979 0.992989448 0.994122661

The answer is 23 people - probably much less than you thought!!

