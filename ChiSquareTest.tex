%========================================================================================%
\begin{frame}[fragile]
\frametitle{Chi-Square Test}
In the built-in data set survey, the Smoke column records the students smoking habit, while the Exer column records their exercise level. The allowed values in Smoke are "Heavy", "Regul" (regularly), "Occas" (occasionally) and "Never". As for Exer, they are "Freq" (frequently), "Some" and "None".
We can tally the students smoking habit against the exercise level with the table function in R. 

The result is called the contingency table of the two variables.
\end{frame}
%========================================================================================%
\begin{frame}[fragile]
\frametitle{Chi-Square Test}
Solution
> library(MASS)       # load the MASS package 
> tbl = table(survey$Smoke, survey$Exer) 
> tbl                 # the contingency table 
 
        Freq None Some 
  Heavy    7    1    3 
  Never   87   18   84 
  Occas   12    3    4 
  Regul    9    1    7
Problem
Test the hypothesis whether the students smoking habit is independent of their exercise level at .05 significance level.
\end{frame}
%========================================================================================%
\begin{frame}[fragile]
\frametitle{Chi-Square Test}
Solution
We apply the chisq.test function to the contingency table tbl, and found the p-value to be 0.4828.

\begin{verbatim}
> chisq.test(tbl) 
 
        Pearson’s Chi-squared test 
 
data:  table(survey$Smoke, survey$Exer) 
X-squared = 5.4885, df = 6, p-value = 0.4828 
 
Warning message: 
In chisq.test(table(survey$Smoke, survey$Exer)) : 
  Chi-squared approximation may be incorrect
%========================================================================================%
\begin{frame}[fragile]
\frametitle{Chi-Square Test}
Solution
Answer
As the p-value 0.4828 is greater than the .05 significance level, we do not reject the null hypothesis that the smoking habit is independent of the exercise level of the students.
