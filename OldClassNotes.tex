Contents
Introduction to R	3
History of R	3
The R Programming Language	3
Using the Help functions	3
Packages	3
Objects	4
Probability distributions	4
Other Mathematical Functions	5
Complex numbers	5
Trigonometric  Functions	5
Useful R commands for linear algebra	5
Numerical methods	6
More on Matrices	6
Descriptive Statistics and Basic Graphical Methods	8
Measures of Centrality and Dispersion	8
Histograms	9
Boxplots	9
Stem and Leaf Plots	9
Bivariate Data	9
Simple Scatterplots, Correlation and Covariance	9
Anscombe’s Quartet	9
The uniform distribution	10
Simulating a dice experiment	10
The "apply" family of functions	10
Discrete Probability Distributions	10
Continuous Probability Distributions]	11
The Normal distribution	11
Iteration Loops	12
Central Limit Theorem	12
Gambler’s Ruin	13
Markov Chains	14

 
Introduction to R
History of R

The R Programming Language
R is a flexible language that is object oriented and thus allows the manipulation of a complex
data structures in a condensed and efficient manner.
 
R's graphical abilities are also remarkable with possible interfacing with text processors such as
Latex with the package sweave.
R offers the addtional advantage of being a free and opensource system under the GNU general
public licence.

R is primarily a statistical language.
R can be installed free of charge from www.r-project.org

An online guide "An Introduction to R" can be access by typing
help.start() at the command prompt to access this.

Defining variables
R is case sensitive.
A convention is to use define a variable name with a capital letter.
This reduces the chance of overwriting inbuild R functions, which are usually written in lowercase letters.


Functions in R can be treated as "objects" that can be manipulated or used recursively.

R shares many aspects with both Object orietnated and Functional programming langguages.
all data in R is stored an objects, which have a range of "methods" available.
The "class" of an object can be found using the class() function.
Using the Help functions
Embedded help commands "help()" and "help.search()" are good starting points to gather information. 
Note that "help.search()" opens a web browser linked to the local manual pages.
 
 
%--------------------------------------------------------------------------------------------------------------------------------------%
Probability distributions
The discrete uniform distribution
parameters: min , max.
The default values are 0 and 1.
 
Other Mathematical Functions

Complex numbers
x = -1 ;  sqrt(x)  ;  str(x) ; 	# variable is defined as numeric, not complex.
y = -1 +0i ;  sqrt(y)  ;  str(y) ;    	#variable is defined as complex .
Trigonometric  Functions
pi				#returns the value of pi to six decimal places
sin(3.5*pi)			# correct answer is -1
cos(3.5*pi)			# correct answer is zero
Useful R commands for linear algebra
# Linear Algebra
# construct a three-by-three matrix. Values entered by column.
Mat1 = matrix(c(1,2,2,0,5,0,1,2,3),ncol=3)
?matrix				# access the help function entry for “matrix”.
Mat1				#print the matrix on the screen.
dim(Mat1)			#determine the dimensions.
Mat1 = t(Mat1) 		#transpose the matrix permanently.
det(Mat1)			#calculate the determinant. Is it non-zero?
solve(Mat1) 			#calculate the inverse.
eigen(Mat1)			#compute eigenvalues and eigenvectors .
names(eigen(Mat1))		# determine the name.
eigen(Mat1)$values 		#access the eigenvalues.

 
Numerical methods
Matrrix “Mat2” is a positive definite symmetric matrix.
Dvls = diag(Mat1)		# extract vector of the diagonal elements of 'Mat1'
Dvls				#print to screen
as.matrix(Dvls)			# display the vector in matrix form (3R1C)
t(as.matrix(Dvls)		# transpose of above	(1R3C)
Idty = diag(c(1,1,1))		# create the identity matrix
Idty				#print to screen
Dgnl = diag(diag(Mat1))	# create a 'D' matrix 	(courework: diagonalization)
Dgnl				
Dvls * Idty			# alternative method with scalar multiplication.
				# N.B.  Dvls is a vector, not a matrix.
Dgnl.inv = solve(Dgnl)		# invert the D matrix
Dgnl.inv		
Rmdr = Mat1 - Dgnl		# Compute the "Remainder" matrix (elementwise subtraction)
# construct a positive definite symmetric three-by-three matrix
Mat2 = matrix(c(1,3,1,3,13,7,1,7,6),ncol=3)
chol(Mat2)			#Cholesky decomposition of Matrix A
More on Matrices
Note that the following commands are useful for Experimental Design.
rowMeans(Mat1) 			#Returns vector of row means. 
rowSums(Mat1) 			#Returns vector of row sums.  
colMeans(Mat1) 			#Returns vector of column means.  
colSums(Mat1) 				#Returns vector of coumn means.  
If x and y are matrices then the covariances (or correlations) between the columns of x and the columns of y are computed.


var( Mat1[,1] )			# determine the variance of the first column 
var ( Mat1[,2] )			# determine the variance of the second column
var ( Mat1[3,] )			# determine the variance of the third row
cov ( Mat1[,1], Mat1[,2] )	# covariance of the first two columsn
var ( Mat1)			# variance covariance matrix of all columns
VCmat=var( Mat1)		# Save as matrix “VCmat”
cor ( Mat1)			# correlation matrix of all columns
cov2cor( VCmat)		#convert a VC matrix to a correlation matrix

 
Descriptive Statistics and Basic Graphical Methods
Measures of Centrality and Dispersion
# Anscombe’s  Quartet
X1 = c(
Y1 =
 

Anscombe’s Quartet 
The uniform distribution

Simulating a dice experiment
Dice = runif(50,min=0,max=6)	
Dice				#print the outcomes to screen
floor(Dice)			#using the floor function
ceiling(Dice)			# using the ceiling function
Dice=ceiling(Dice)		#discretize the outcomes.
mean(Dice)			# compute the mean of the outcomes (expected value).
sd(Dice)			#compute the standard deviation of the outcomes.
# More compact alternative, using the floor function. Note the changed parameters.
Dice2=floor(runif(50,min=1,max=7))	

Questions
The expected value of a dice throw is 3.5. Is the sample mean close to this value?

The "apply" family of functions
 The Apply family of functions are 
Discrete Probability Distributions

The two most accessible discrete distributions are the binomial and poisson distributions
The binomial distribution is characterized by the number of trials, which in R is denoted as ‘size’ rather than ‘n’, and the probability of success ‘prob’.
rbinom(n=5,size=100,p=0.25)		#generate five random numbers

The Poisson distribution is characterized the by the arrival rate ‘lambda’.
rpois(n=5,lambda=4)			#generate five random numbers
 
Simple population study.

Suppose a small island has population 1,000 at the start of a decade. The birth rate on this island is expected to 25 births per annum, while there is on average 10 deaths.  Forecast the population after five years.

Base = 1000
Births =rpois(5,25)
Deaths=rpois(5,10)
Yrly.Incr =Births - Deaths
Increase =cumsum(Yrly.Incr)
Popn = Base + Births +Deaths

Continuous Probability Distributions]

The continuous uniform distribution is commonly used in simulation.
The Normal distribution
The normal distribution is perhaps the most w
rnorm(n=15)				#15 random numbers, mean  = 0 , std. deviation = 1
rnorm(n=15,mean= 17)			#set the mean to 17
rnorm(n=15,mean= 17,sd=4)		#set the standard deviation to 4
rnorm(15,17,4)				#argument matching : default positions


The exponential distribution


 
Iteration Loops
Central Limit Theorem
Recall the Dice experiment in week 8.

N=100				#number of loops
Avgs=numeric(N)		#array “Avgs” store the sample means
for( i in 1:N)
	{	Dice=floor(runif(50,min=1,max=7));	Avgs[i]=mean(Dice);
	}
Avgs				#print Avgs dataset to screen

The Central limit theorem states that.
The “Dice” distribution is a discrete uniform distribution. However 
mean(Avgs)			#compute the mean. Is it roughly what we are expecting?
qqnorm(Avgs)			#draws a QQ plot that is used to check for normality.
qqline(Avgs)			#adds trend line to QQplot.
shapiro.test(Avgs)		#Shapiro Wilk test. Normality is assumed if p-value > 0.05.
			
 
Gambler’s Ruin
In anticipation of future material on course MS4217.
 
