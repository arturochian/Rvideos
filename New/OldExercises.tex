

MS4024 (R) - Worksheet 1

 

 Exercise 1: Evaluate the following:

 
1.
the area of a circle of radius 2             

2.
the natural log of  100                           

3.
the log of 100 to the base 10   

4.
the exponential of 10                           

5.
the square of 12                                         

6.
the sine of 1.5 pi                           

7.
the absolute value of  -9             

8.
the square root of 110             

9.
the factorial of 6


 

How many ways are there of choosing two items from a batch of ten items?

 

Exercise 2: Numeric Vectors

 

1.      Generate a sequence of values from 1 to 10

2.      Create a vector ‘x’ of those values .

3.      Generate a sequence of values from 5 to 8 incrementing by 0.5

4.      Create a vector ‘y’ with values from 10 to 1 (i.e. in decreasing order).

5.      Create a vector w <- pi + sqrt(x)/2.

6.      Round the values of ‘w’ to two decimal places.

7.      Round the values of ‘w’ down into integer values.

8.      Round the values of ‘w’ up into integer values.

 

Exercise 3 Other vector types

 
1.
Create a character vector of the following names: Alan, Brian, Cathal, Dermot and Enda

2.
Create a logical vector ‘logic’ of the following values: True False False True True

3.
Create a binary equivlanent to ‘logic’.

4.
Generate a vector of values 1 to 5.  Determine the letters that correspond to those values.

5.
Generate a vector of values 5 to 9.  Determine the months that correspond to those values.


 

Exercise 4

 

Suppose you keep track of your mileage each time you go to the petrol station. At your last 8 fill-ups the mileage was

 

65311 65624 65908 66219 66499 66821 67145 67447

 

 Create a vector ‘miles’ from this data. Use the function diff() on the data. What does it give?

 

> miles = c(65311, 65624, 65908, 66219, 66499, 66821, 67145, 67447)

> x = diff(miles)

 

You should see the number of miles between fill-ups. Find the maximum number of, the average number of miles and the minimum number of miles between fill-ups.

 

Exercise 5

 

The following data gives, for each amount by which an elastic band is stretched over

the end of a ruler, the distance that the band moved when released.

 














Stretch (mm)
 

46
 

54
 

48
 

50
 

44
 

42
 

52
 



Distance (cm)
 

148
 

182
 

178
 

166
 

109
 

141
 

166
 

 
1.
Create a data frame called ‘elasticband’. (use data.frame() command)

2.
Create summary statistics for the data frame.   (use summary() command)

3.
Create a boxplot of the distance.   (use boxplot() command)

4.
What is the correlation between stretch and distance (use cor() command)

5.
What is the covariance?  (use cov() command)

6.
Create a plot of distance versus stretch.  (use plot() command)             

7.
Create a simple histogram of the distance. (use hist() command)


 

Exercise 6

 

Your cell phone bill varies from month to month. Suppose your year has the following monthly amounts

 

46 33 39 37 46 30 48 32 49 35 30 48

 

Enter this data into a variable called ‘bill’. Use the sum() command to find the amount you spent this year on the cell phone.

 

1.      What is the smallest amount you spent in a month?

2.      What is the largest?

3.      What is the inter quartile range of this dataset?  (use IQR() command)

4.      Which month (i.e. index number ) has the largest value?

5.      How many months was the amount greater than €40?

6.      What percentage was this?

 

 Exercise 7

 

The Whiteside data is stored in the MASS package.

 

> library(MASS)

 
1.
Get a description of the Whiteside data.

2.
What are the names of the columns in the Whiteside data?

3.
Create summary statistics for the Whiteside data.

4.
Plot gas consumption versus temperature. Use plot symbol 15, and title the plot "Whiteside’s Gas consumption” .


 

Exercise 8

 

Eight patients were asked to rate their pain from 0 to 3, with 0 representing `no pain', 1 representing `mild' pain, 2 representing `medium' pain and 3 representing `severe' pain. The following results were obtained:

 















Patient
 

1
 

2
 

3
 

4
 

5
 

6
 

7
 

8
 



Pain level
 

0
 

3
 

3
 

1
 

2
 

1
 

0
 

2
 

 

Create a factor ‘fpain’ to represent the above data.

 

Exercise 9

 

Construct a matrix A with values 2, 7, 2 in row 1, values 5, 1, 5 in row 2 and values 5, 6, 7 in row 3, i.e. a 3×3 matrix. Also construct a  3×1 matrix B with values 1,5,7. 

 
1.
What is the determinant of A?

2.
What is the inverse of A?

3.
Multiply A by B

4.
Solve a system of linear equations Ax=B

5.
Construct a matrix C, the transpose of B.

6.
Combine A and B into a new matrix D using cbind().

7.
Combine A and C into a new matrix H using rbind().

8.
Determine the dimensions of D and H using dim() function.

9.
What are the eigenvalues of A?


  

Exercise 10

 

Plot the following functions ( between -2π and 2π). Include the X and Y axes on your plot.

 

1.    Cosine                              ( use line type 7, colour  red)

2.    Sine                                  ( use line type 8, colour blue)

3.    Exponential                      ( use line type 9, colour green)

4.     Absolute value                 ( use line type 10, colour pink)

 

Exercise 11

 

The ‘Iris’ data is a resident dataset in R.

 
1.
What are the names of each column of this data set?

2.
What type of data does each column contain?

3.
Which column has the highest pair-wise correlation with the first column (excluding the final non-numerical column)?


 

Exercise 12

 

The resident data set ‘mtcars’ contains information about cars from a 1974 Motor Trend issue. Load the data set and answer the following questions.

>data(mtcars)

 

1. What are the variable names?

2. What is the maximum mpg?

3. Which car has this?

4. What are the first 5 cars listed?

5. What horsepower (hp) does the "Valiant" have?

 

 

Exercise 13

 

Create 4 vectors Year, Gender, mean_weight, and mean_height from the data in the table below.

 













Year
 

1980
 

1988
 

1996
 

1998
 

2000
 

2002
 



Gender
 

M
 

M
 

F
 

F
 

M
 

M
 



Mean weight
 

71.5
 

72.1
 

73.7
 

74.3
 

75.2
 

74.7
 



Mean height
 

179.3
 

179.9
 

180
 

180.1
 

180.3
 

180.4
 

 

Create a list called ‘mylist’ consisting of the above vectors. (Give each component of the list a name.)

 

Use 3 different ways to access the 4th element of the list.

 

 

Exercise 14

evalut the following

 

Exercise 15

 

In the library MASS is a dataset UScereal which contains information about popular breakfast cereals.

 

Attach the data set as follows

> library('MASS')

> data('UScereal')

> attach(UScereal)

> names(UScereal)         # to see the names

 

Now, investigate the following relationships, and make comments on what you see. You can use tables, barplots, scatterplots etc. to do your investigation.

 

1. The relationship between manufacturer and shelf

2. The relationship between fat and vitamins

3. The relationship between fat and shelf

4. The relationship between carbohydrates and sugars

5. The relationship between fibre and manufacturer

6. The relationship between sodium and sugars

 

Exercise 16

 

Create a data frame called ‘club.points’ with the following data. (see clubpoints.R )

 











First name
 

Last name
 

Age
 

Gender
 

Points
 



Louise
 

Acheson
 

35
 

F
 

278
 



John
 

McHale
 

28
 

M
 

242
 



Sean
 

Doyle
 

34
 

M
 

312
 



Kevin
 

Comerford
 

72
 

M
 

740
 



Alison
 

Reynolds
 

19
 

F
 

147
 



Rebecca
 

Fortune
 

24
 

F
 

195
 



Joanne
 

Smyth
 

42
 

F
 

351
 



Andrea
 

Healy
 

22
 

F
 

183
 



David
 

Keenan
 

30
 

M
 

201
 

 
1.
Store the points for every person into a vector called pts, then calculate the average number of points received. (Hint use mean() function).

2.
Store the data for the females only into a data frame called fpoints.

3.
The age for Sean Doyle was entered incorrectly. Change his age to 36.

4.
Determine the maximum age of the males.

5.
Extract the data for people with more than 200 points and are over the age of 30.


 

Exercise 17

 

If the lengths of all three sides (a,b,c) of a triangle are known, then the area is given by Heron's Formula.

Area = sqrt(s × (s-a) × (s-b) × (s-c))

 

where s = (a + b + c) / 2

 

Write a function that calculates the area of a triangle when the lengths of three sides are given as arguments.

 

Write a function that calculates the internal angles of the triangle, using the cosine rule

Cosine rule: a2 = b2 + c2- 2bc(cosA)

 

A is the angle opposite side ‘a’ and between sides ‘b’ and ‘c’

 

Exercise 18

 
1.
Generate three set of 15 randomly generated numbers, ‘x’, ‘y’ and ‘z’.

2.
Make a single scatterplot of ‘y’ against ‘x’               (plot symbol 16 , colour red)

3.
Set the symbol size to 1.1                                           (cex =1.1)

4.
Make another for ‘z’ against ‘x’                             (plot symbol 18, colour blue)

5.
Combine the two scatterplots into the same plot.

6.
Add a vertical line (type 5) indicating the mean value of ‘y’ (colour: red)

7.
Add another for the mean value of ‘z’                (colour: blue).

8.
Add a line (type 7) with intercept 1 and slope 0.5.

9.
Title the plot ‘Scatterplot’.

10.
Title the axes ‘horizontal’ and ‘vertical’.

11.
Add a legend to the plot.


 

Exercise 19

 

Construct a vector of the 0.9, 0.95, 0.975 and 0.99 quantiles of the standard normal distribution.

 

Construct a vector of cumulative distribution functions of the followingvalues

1, 1.28, 1.68, 1.96.

 

Exercise 20

 

Generate a normally distributed random sample of size 300. What is the mean and standard deviation of this sample? What are the 0.9, 0.95, 0.975 and 0.99 quantiles quantiles of this sample? Compare both sets of results.

 

Use R to determine the 90th, 95th and 99th quantiles of a normal distribution, with mean 100 and standard deviation 15.

 

Exercise 21

 

IQ scores are assumed to have a normal distribution with mean 100 and standard deviation 15.

 
1.
What IQ would you have if you were in the 80th percentile?

2.
Estimate the threshold for the top 10 percent?

3.
What is the probability of having an IQ above 142?

4.
What is the probability of having an IQ below 97?


 

 

Exercise 22

 

Use the uniform distribution to simulate 100 throws of two dice. The outcome is the combined values of both dice. Use the appropriate R command to discretize values.

 
1.
What is the mean and standard deviation of the outcomes?

2.
Make a stem-and-leaf plot of the outcomes.

3.
Make a histogram of the outcomes. (hint: use breaks =seq(1.5,12.5))


 

Exercise 23

 

Add up all the numbers for 1 to 50

∙       Using the sum() command.

∙       using a ‘for’ loop.

 

Exercise 24

Create a matrix that contains the pair-wise correlation coefficient of the numeric columns of the ‘Iris’ data. Use nested ‘for’ loops.

 

Exercise 25

Generate a vector of randomly generated numbers (Normal, mean 10, standard deviation 2). Using ‘for’ loops, replace each element that is greater than 11 by the value 11.

 

 

