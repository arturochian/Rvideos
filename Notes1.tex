


Creating and Managing Data Objects


The assignment operator

The concatenate function

data entry

Spreadsheet Style Interface

Vectors

Lists

Output from many important statistical functions are structured as lists.

Arrays

Matrices important in advanced computational disciplines like Machine Learning.


The class function

The str function

 

The summary function


%---------------------------------------------------------------------------- %
\newpage

\section{Programming Control Statements}

\begin{itemize}
\item IF and ELSE statements
\item FOR loop
\item WHILE loop
\item Break statement
\end{itemize}

\begin{itemize}
\item Packages and Taskviews
\begin{itemize}
\item Packages
\item Comprehensive \texttt{R} Archive Network (CRAN)
\item Books (e.g. IPSUR and SimpleR)
\item Introduction to R (HTML)
\item Package Vignettes
\item Taskviews
\item Inbuilt data sets
\end{itemize}
%----------------------------------------------------------------%
\item \texttt{R} Community
\begin{itemize}
\item Twitter : One \texttt{R} tip a day
\item \texttt{R} bloggers
\item Google Coding Style Guide
\item Coursera : Computing for Data Analysis (with Roger Peng)
\end{itemize}
%----------------------------------------------------------------%
\item Using Help
\begin{iitemize}
\item \texttt{help(thing)} or alternatively \texttt{?thing}
\item The HTML master \texttt{help.start()}
\item Using the \texttt{apropos()} function ( try it out for using a command for \textbf{\texttt{correlation}})
\end{itemize}
%----------------------------------------------------------------%
\item Graphics : One of \texttt{R}'s strengths, but not on today's agenda
\begin{itemize}
\item base
\item ggplot2 by Hadley Wickham
\item lattice
\item grid by Paul Murrell
\end{itemize}
%----------------------------------------------------------------%
\item Writing Functions (the \texttt{function()} command)
\item Working Directories (\texttt{getwd()} and \texttt{setwd()}})
\item The \texttt{source()} command
\item Simulation Studies
\begin{itemize}
\item Gambler's Ruin
\item Gambler's Fallacy
\item Monty Hall Problem
\end{itemize}
\end{itemize}



\newpage
\section{Matrices with \texttt{R}}

Matrices are a very useful data structure. 
An inputted vector can be transformed into matrix form before further constructing it as a vector.

Suppose the following vector was scanned in as a vector, but intended as a data frame.
\begin{framed}
\begin{verbatim}
2 1.15 
3 1.56
5
2

DAT = c(2
\end{verbatim}
\end{framed}
%-------------------------------%
\section{Reading in Data}



%-----------------------------------------------------------------------------------------------------------%


%---------------------------------------------------------------------- %
\section{Monty Hall Problem}
Suppose that there are three closed doors on the set of the  \textbf{\emph{Let's Make a Deal}}, a TV game show presented by Monty Hall. Behind one of these doors is a car; behind the other two are goats. The contestant does not know where the car is, but Monty Hall does.

The contestant selects a door, but not the outcome is not immediately evident. Monty opens one of the remaining ``wrong" doors. If the contestant has already chosen the correct door, Monty is equally likely to open either of the two remaining doors.

After Monty has shown a goat behind the door that he opens, the contestant is always given the option to switch doors. What is the probability of winning the car if she stays with her first choice? What if she decides to switch?



\subsection{Implementation (part 1)}
We have 3 doors to choose from, so we will define a numeric sequence between 1 and 3. The command \texttt{\textbf{sample(,n)}} takes a sample of size n from a specified set of values. Here we just want to select one door to be our "correct door" and another to be ``selected" door (i.e. the contestant selects)

These events are independent. We will perform the selection for both doors separately, but this can be implemented in one command.


\begin{framed}
\begin{verbatim}
Doors = 1:3
Correct = sample(Doors,1)
Choice = sample(Doors,1)
\end{verbatim}
\end{framed}

A wrong door must be selected to be opened. The door selected by the contestant can't be chosen. First let us select the doors that must stay closed, then find the ones we can choose from to open

\begin{framed}
\begin{verbatim}
StayClosed = union(Correct, Choice)
CanOpen = setdiff(Doors, StayClosed)
\end{verbatim}
\end{framed}

The \texttt{R} command \texttt{\textbf{sample()}} poses an interesting problem. Lets look at the help file to see what it is.

Create a single element vector (let's call it \texttt{\textbf{Vec}}, with that single element being 4. Now sample a single value from that data set. You may expect to get 4 each time, but \texttt{R} doesn’t work that way in this instance.

\begin{framed}
\begin{verbatim}
Vec=c(4)
sample(Vec,1)
\end{verbatim}
\end{framed}

\begin{verbatim}
> sample(Vec,1)
[1] 3
> sample(Vec,1)
[1] 4
> sample(Vec,1)
[1] 1
\end{verbatim}
To overcome this we need some control statements. Now we have the doors we are able to open.

\begin{framed}
\begin{verbatim}
if(length(CanOpen)>1)
{
Open = sample(CanOpen,1)
}else {Open=CanOpen}
\end{verbatim}
\end{framed}


\begin{framed}
\begin{verbatim}
NotOpen = setdiff(Doors,Open)

Stick = Choice
    #The previous statement is to aid the narrative.

Switch = setdiff(NotOpen,Choice)

# Was sticking the right decision? Or was switching?
# The following logicial statements  will tell us.

Stick==Choice
Switch==Choice
\end{verbatim}
\end{framed}


\subsection{Writing a Function}
We are going to create a function \texttt{\textbf{MHfunc()}} to help us simulate a solution for the Monty Hall Problem. The function is constructed using \texttt{R} code we have seen already. The output of the function is returned as a data frame, with three columns:
\begin{description}
\item[Correct] : The number of the correct door.
\item[Choice] :  The door that the contestant chose originally, and the door selected if the contestant decided to ``stick".
\item[Switch] : The door selected if the contestant chose to ``switch".
\end{description}
Necessarily the Correct option must correspond with a value in one of the other two columns.
\begin{framed}
\begin{verbatim}
MHfunc <- function(numdoors=3){
 Doors = 1:numdoors
 Correct = sample(Doors,1)
 Choice = sample(Doors,1)

 StayClosed = union(Correct, Choice)
 CanOpen = setdiff(Doors, StayClosed)

 if(length(CanOpen)>1)
    {
    Open = sample(CanOpen,1) #Problem here
    }else {Open=CanOpen}

 NotOpen = setdiff(Doors,Open)
 Switch = setdiff(NotOpen,Choice)


 MHoutput=c(Correct,Choice,Switch)
 names(MHoutput)= c("Correct","Choice","Switch")
 return(MHoutput)
}
MHfunc()

\end{verbatim}
\end{framed}

No arguments are needed to be passed to the function. The output should appear something like this:
\begin{verbatim}
> MHfunc()
  Correct Choice Switch
1       1      2      1
> MHfunc()
  Correct Choice Switch
1       1      3      1
> MHfunc()
  Correct Choice Switch
1       3      3      1
\end{verbatim}

\subsection{Transforming logical values}

The \texttt{R} command \texttt{\textbf{as.numeric()}} can be use to convert logical values, such as TRUE or FALSE into binary form (i.e. 1 and 0).

\begin{framed}
\begin{verbatim}
LogVec=c(TRUE,FALSE,TRUE)
as.numeric(LogVec)
\end{verbatim}
\end{framed}
\begin{verbatim}
> output = MHfunc()
> output
Correct  Choice  Switch
      1       2       1
> output[1]
Correct
      1
> output[2]
Choice
     2
> output[1]==output[2]
Correct
  FALSE
> output[1]==output[3]
Correct
   TRUE
> as.numeric(output[1]==output[2])
[1] 0
> as.numeric(output[1]==output[3])
[1] 1
   >
\end{verbatim}

\subsection{A Simulation Study of the Monty Hall Problem}

Let us perform a simulation study of the Monty Hall problem. We are putting the code we have written already, placed in a \texttt{\textbf{for}} loop, although we alter the ending to make for more readable output.

\begin{framed}
\begin{verbatim}
StickCount = 0
SwitchCount = 0
M=1000

for(i in 1:M)
 {
 output=MHfunc()
 Correct=output[1]
 Choice=output[2]
 Switch=output[3]
 # Install counters for both “count variables”

 StickCount = StickCount + as.numeric(Choice  ==  Correct)
 SwitchCount = SwitchCount + as.numeric(Switch  ==  Correct)
}
StickCount
SwitchCount
\end{verbatim}
\end{framed}

The output of this program would look like this. The proportion is approximately 2 to 1 in favour of the switching option.
\begin{verbatim}
> StickCount
[1] 323
> SwitchCount
[1] 677
\end{verbatim}
\newpage
%---------------------------------------------------------------%
\section{Gambler's Ruin}

Consider a gambler who starts with an initial fortune of \$1 and then on each successive gamble
either wins \$1 or loses \$1 independent of the past with probabilities p and q = 1-p respectively.

Suppose the gambler has a starting kitty of A.
This gamblers places bets with the �Banker�, who has an initial fortune B. We will look at the game from the perspective of the gambler only.
The Banker is, by convention, the richer of the two.

\begin{itemize}
\item Probability of successful gamble for gambler : p
\item Probability of unsuccessful gamble for gambler : q 	(where q =  1 - p )
\item Ratio of success probability to failure success:	$s = p / q$
\item Conventionally the game is biased in favour of the Banker (i.e. $q>p$ and $s<1$)
\end{itemize}

Let $R_n$ denote the Gambler�s total fortune after the $n-$th gamble.

If the Gambler wins the first game, his wealth becomes $R_n =A+1$.
If he loses the first gamble, his wealth becomes $R_n = A-1$.
The entire sum of money at stake is the �Jackpot� i.e.   $A+B$.
The game ends when the Gambler wins the Jackpot ($R_n = A+B$) or loses everything ($R_n = 0$).


\subsection{Simulation a Single Gamble}


To simulate one single bet, compute a single random number between 0 and 1.
\begin{framed}
\begin{verbatim}
runif(1)
\end{verbatim}
\end{framed}

Lets assume that the game is biased in favour of the Banker
p = 0.45 , q = 0.55.
If the number is less than 0.45, the gamble wins. Otherwise the Banker wins.

\begin{verbatim}
> runif(1)
[1] 0.1251274
>#Gambler Loses
>
> runif(1)
[1] 0.754075
>#Gambler wins
>
> runif(1)
[1] 0.2132148
>#Gambler Loses
>
> runif(1)
[1] 0.8306269
\end{verbatim}

%----------------------------------------------------------------------%
\newpage
\begin{itemize}
\item Let A be the Gambler's Kitty at the start of the gambling.
\item Let B be the Banker's Wealth.
\item The probability of A winning a gamble is p.
\item The vector $R_n$ records the gambler's worth on an ongoing basis. At the start, The first value is A.
\end{itemize}
%----------------------------------------------------------------------%

\begin{framed}
\begin{verbatim}

A=20;B=100;p=0.47
Rn=c(A)

probval = runif(1)

if (probval < p)
 {
 A = A+1; B =B-1
 }else{A=A-1;B=B+1}

#Save the values from each bet
Rn=c(Rn,A)
\end{verbatim}
\end{framed}
\newpage
Should the Gambler win the entire jackpot (A+B). The game would also cease. We include a \textbf{\texttt{break}} statement to stop the loop if the gambler wins the entire jackpot. A break statement will stop a loop if a certain logical condition is met.

\begin{framed}
\begin{verbatim}
A=20;B=100;p=0.47
Rn=c(A)
Total=A+B

while(A>0)
  {
  UnifVal=runif(1)

  if(UnifVal <= p)
     {
     A = A+1; B =B-1
     }else{A=A-1;B=B+1}
  Rn=c(Rn,A)
  if(A==Total){break}
  }
\end{verbatim}
\end{framed}
We can construct a plot to depict the gambler's ongoing fortunes in the game. We use a \texttt{for} loop, that implements the game, recording the duration of the game each time. The duration of each game is the dimension of the \texttt{Rn} vector, i.e. \texttt{length(Rn)}.


\begin{framed}
\begin{verbatim}
A.ini=20;B=100;p=0.47
M=1000
RnDist=numeric()
Total=A+B

for (i in 1:M)
    {
    Rn=numeric()
    Rn[1]=A
    A=A.ini
    while(A>0)
        {
       
        UnifVal=runif(1)

        if(UnifVal <= p)
            {
            A = A+1; B =B-1
            }else{A=A-1;B=B+1}
        Rn=c(Rn,A)
        if(A==Total){break}
        }
    RnDist=c(RnDist,length(Rn))
    }
    

\end{verbatim}
\end{framed}
\subsection{Distribution of Durations}
We can use for loop to get a sense of the duration (i.e. the number of rounds a game will last).
\end{document}





\subsection*{About Revolution Analytics}
Revolution Analytics is a statistical software company focused on developing "open-core" versions of the free and open source software R for enterprise, academic and analytics customers. 
\\
\\
\noindent Revolution Analytics was founded in 2007 as REvolution Computing providing support and services for R in a model similar to Red Hat's approach with Linux in the 1990s as well as bolt-on additions for parallel processing. 
\\
\\

\noindent Their core product, Revolution R, would be offered free to academic users and their commercial software would focus on big data, large scale multiprocessor (or "high performance") computing, and multi-core functionality.
%------------------------------------------------
\newpage
\section{Miscellaneous Commands}
% http://www.genetics.med.ed.ac.uk/resources/r/ListOfRcommands.pdf

\begin{framed}
\begin{verbatim}
x = c(3,5,8,9,11,14,17,22)
mean(x) ; median(x) ; sum(x)
range(x);min(x);max(x)

> x = c(3,5,8,9,11,14,17,22)
> mean(x) ; median(x) ; sum(x)
[1] 11.125
[1] 10
[1] 89
> range(x);min(x);max(x)
[1]  3 22
[1] 3
[1] 22
> diff(x)
[1] 2 3 1 2 3 3 5
> 
\end{verbatim}
\end{framed}




%-----------------------------------------------------------%
\section{Character Manipulation}
R is usually thought of as a language designed for numerical computation,
it contains a full complement of functions which can manipulate character
data.

\begin{itemize}
\item The \texttt{nchar()} function can be used to find the number of characters in a character value.
\item Character values will be displayed when they are passed to the  \texttt{print()} function.
\item The \texttt{cat()}  function will combine
character values and print them to the screen or a file directly. The \texttt{cat()}
function coerces its arguments to character values, then concatenates and displays
them. This makes the function ideal for printing messages and warnings
from inside of functions.
\end{itemize}

%-----------------------------------------------------------%
\subsection{Regular Expressions in \texttt{R}}
Regular expressions are a method of expressing patterns in character values
which can then be used to extract parts of strings or to modify those strings in some way. Regular expressions are supported in the \texttt{R} functions \texttt{strsplit()},
\texttt{grep()}, \texttt{sub()}, and \texttt{gsub()}, as well as in the\texttt{ regexp()} and \texttt{gregexpr()} functions which
are the main tools for working with regular expressions in \texttt{R}.


\newpage
\section{Some Well Known Probability Concepts}
\subsection{The Birthday Distribution}
To familiarise ourselves with probability with \texttt{R}, let's consider the \textbf{\textit{Birthday Distribution}}.

\begin{framed}
\begin{verbatim}
> 1/365
[1] 0.002739726
>
> pbirthday(2)
[1] 0.002739726
>
> pbirthday(23)
[1] 0.5072972
\end{verbatim}
\end{framed}
%-----------------------------------------------------------%
\subsection{The Uniform Distribuion}
The default parameters of the \texttt{runif()} function is \texttt{min=0} and \texttt{max=0}.
\begin{framed}
\begin{verbatim}
> runif(1)
[1] 0.6312415
>
> runif(4)
[1] 0.02355764 0.43060817 0.92650927 0.81789815
\end{verbatim}
\end{framed}

Other values for \texttt{min} and \texttt{max} can be specified.
\begin{framed}
\begin{verbatim}
> runif(4,min=0,max=10)
[1] 9.534040 7.575022 5.495045 4.544155
>
> runif(4,min=0,max=6)
[1] 3.388653 4.477615 5.991367 2.204642
\end{verbatim}
\end{framed}
%-----------------------------------------------------------%
\subsection{Simulating a Coin Flip}

\begin{framed}
\begin{verbatim}
> X =  runif(4,min=0,max=2)
>
> X
[1] 1.5856218 1.0754732 0.1659964 0.9007960
>
> floor(X)
[1] 1 1 0 0  # Two heads followed by two tails.
\end{verbatim}
\end{framed}


(Equally, we can use this to simulate red of black for BlackJack)
%-----------------------------------------------------------%
\subsection{Sampling}
The \texttt{sample()} function is used to sample a specified number
of items from a vector.\\

\bigskip
\noindent \textbf{Sequences}: A sequence of integers is generated usng the colon operator. For example \texttt{1:6} generates a sequence of consecutive integers for 1 to 6.

\subsubsection{Sampling without Replacement}
The default setting for the \texttt{sample()} function is the
\textbf{\textit{sampling without replacement}}; once an item has been selected, it can not be selected again. A commonly encountered example of sampling without replacement is choosing lottery numbers.
\begin{framed}
\begin{verbatim}
> 1:42
 [1]  1  2  3  4  5  6  7  8  9 10 11 12 13 14 15 16 17 18 19 20
[21] 21 22 23 24 25 26 27 28 29 30 31 32 33 34 35 36 37 38 39 40
[41] 41 42
> 
> sample(1:42,6)  #pick 6 integers between 1 and 42
 [1]  9 26 10 13 32 14
\end{verbatim}
\end{framed}


\subsubsection{Sampling with Replacement}


\begin{framed}
\begin{verbatim}
> 1:6
[1] 1 2 3 4 5 6
> 
> sample(1:6,10,replace=TRUE)
 [1] 2 3 1 3 2 5 6 6 3 2
\end{verbatim}
\end{framed}

\newpage
\section{Dice Rolls and $p-$values}
Consider an experiment where a die is rolled 100 times, with the sum of the 100 rolls being recorded.
\subsection{Expected value}
\begin{itemize}
\item The highest value that is mathematically possible is 600.
\item The lowest value that is mathematically possible is 100.
\item Realistically we expect a value that is half way between the two (i.e. 350)
\end{itemize}
\subsection{Implementation}
\begin{framed}
\begin{verbatim}
> sum(sample(1:6,100,replace=TRUE))
[1] 356
> sum(sample(1:6,100,replace=TRUE))
[1] 382
> sum(sample(1:6,100,replace=TRUE))
[1] 346
> sum(sample(1:6,100,replace=TRUE))
[1] 352
\end{verbatim}
\end{framed}
%-----------------------------------------------------------%


How many times did the Gambler Win?

For the sake of simplicity, lets make both A and B relatively small.

\newpage
\section{Black Jack and the Gambler's Fallacy}

\begin{quote}
The most famous example of the gambler's fallacy occurred in a game 
of roulette at the Monte Carlo Casino on August 18, 1913, when the ball fell in black 26 times in a row. This was an extremely uncommon occurrence, although no more nor less common than any of the other 67,108,863 sequences of 26 red or black. Gamblers lost millions of francs betting against black, reasoning incorrectly that the streak was causing an "imbalance" in the randomness of the wheel, and that it had to be followed by a long streak of red.
\end{quote}

%----------------------------------------------------------------%

\begin{itemize}
\item Introduction to \texttt{R}
\item probability with \texttt{R}
\item Introduction to ggplot2
\item Introduction to Time Series
\item GIS with R
\item more om R

\end{itemize}


%----------------------------------------------------------------%


%----------------------------------------------------------------%
\begin{itemize}
\item Statistical Analysis with \texttt{R}
\item Advanced Statistical Analysis with \texttt{R}
\item Data Managements with \texttt{R}
\item Geographical Information Systems with \texttt{R}
\item ggplot2 
\item Rggobi
\end{itemize}
%----------------------------------------------------------------%
\begin{itemize}
\item \textbf{\texttt{chron}}
\item \textbf{\texttt{lubridata}}
\item \textbf{\texttt{reshape}}
\item \textbf{\texttt{plyr}}
\item \textbf{\texttt{qcc}}
\item \textbf{\texttt{tourr}}
\end{itemize}
%----------------------------------------------------------------%
\begin{itemize}
\item Dates and Time
\item POSix data
\item Regular Expressions
\item The \texttt{do.call} function
\item debugging
\end{itemize}
%----------------------------------------------------------------%

\begin{frame}
\frametitle{R Video Title}
\begin{itemize}
\item
\item
\end{itemize}

A random forest is a classifier that is built from nultiple tress, generated by random sampling cases and variables.
Forests are computationally intensive but retain some of the interpretability of trees. There are several parameters that control the algorithm, and there
 are numerous diagnostics output by random forests.

%-----------------------------------------------------------------------------%
\subsection{SWeave and knitr}

Sweave Intro
Why Sweave?
Makes research and analysis reproducible.
All code, analysis, and interpretation are in one document.
Sweave is included in every R install. It's used to create R help pages if you are building a package. 

If you're brand new to Sweave, I recommend that you instead use knitr instead.
 It requires R version 2.14 or better, but it has some advantages, and you need either it or Sweave, not both.

Quick outline of the process
 Create a new folder for this (and every) project. 

Create an Sweave document ending in .Rnw (example below) 

Sweave the file though R to create a latex file 

LaTeX the resulting file to get a pdf. 

Rstudio does 3 and 4 in one click. 

http://www.math.montana.edu/~jimrc/classes/Rseminar/SweaveIntro.html

knitr is a comprehensive package derived from Sweave that includes code formatting, highlighting, caching, fine control of graphics, conditional evaluation, multiple markup formats and other features. 





  

 

MS4024 Week 11a

 

Statistical Methods

R includes a host of statistical methods and tests of hypotheses. We will focus on the most

common ones in this Chapter.

One and Two-sample t-tests

The main function that performs these sorts of tests is t.test(). It yields hypothesis tests

and confidence intervals that are based on the t-distribution. 


 



--------------------------------------------------------------------------------
 

http://www.r-tutor.com/elementary-statistics/hypothesis-testing/two-tailed-test-population-proportion

 

One Sample Inference

Two Sample Inference



--------------------------------------------------------------------------------
 

If we have a single sample we might want to answer several questions:

 

What is the mean value?

Is the mean value significantly different from current theory?(Hypothesis test)

What is the level of uncertainty associated with our estimate of the mean value? (Confidence interval)

 

To ensure that our analysis is correct we need to check for outliers in the data and we also need to check whether the data are normally distributed or not.

  

 

Checking Normality

Graphical methods are often used to check that the data being

analysed are normally distributed. Can use
•
Histogram

•
Boxplot

•
Normal probability (Q-Q) plot


Read the le dat.txt into a data frame called data:

data <- read.table("C:/.../das.txt", header=TRUE)

There is only one column of data, y. Can create a summary of y:

 

 

Checking Normality

Create a boxplot, histogram and Q-Q plot of y:

 

boxplots

boxplot(data$y, ylab="data values")

 

The median line is approximately in the center of the boxplot and the whiskers are approximately the same length. This indicates that

the data are normally distributed. There are no obvious outliers.

 

histograms 

histogram(data$y, xlab="y", main="Histogram of y")

The histogram is approximately symmetric and bell-shaped indicating a normal distribution.

 

qq plots 


> qqnorm(data$y)

> qqline(data$y, lty=2)

All of the points lie in an approximate straight line along the diagonal of the plot indicating a normal distribution. There are no

obvious outliers.

 

Formal tests

A formal test, e.g. the Anderson-Darling test, Kolmogorov-Smirnov test etc. can also be carried out in R.

This requires you to download the 'nortest' package from CRAN.

 

Go to http://cran.r-project.org/ > Packages > nortest.

Click on the nortest_1.0.zip link and click Save. Save the zip file to a suitable location.

The package must be loaded into R by clicking

Packages > Install packages from local zip files and locating nortest.

Once the package has been loaded, we can then use library(nortest) to access the functions.

 

The Anderson-Darling test is carried out using:

 

ad.test(data$y)

Anderson-Darling normality test

data: data$y

A = 0.1634, p-value = 0.9421

 

Remember what this test is doing!!

H0 : The data are normally distributed.

H1 : The data are not normally distributed.

Since the p-value is 0.9421 > 0.05 ) not enough evidence to reject H0 ) the data are normally distributed.

 

The Kolmogorov-Smirnov test is carried out using:

 

Central Limit Theorem

Hypothesis testing and con dence interval construction are based on the Central Limit Theorem.

CLT - see Introductory Data Analysis notes by Dr. Ailish Hannigan.

Can check the CLT using a small simulation example.

We will take 10000 samples of size 5 from data with a uniform distribution and record the means.

When we plot a histogram of the means, should have a normal distribution.

 

means <- numeric(10000)

for(i in 1:10000){

means[i] <- mean(runif(5))

}

hist(means)

 

Hypothesis Testing for the Mean [160]

For large samples (n30) and/or if the population standard deviation () is known, the usual test statistic is given by:


Z =X-S.E(X)

 

where S.E(X) =/n OR S.E(X) = s/(n) and the critical values are looked up in the normal tables.

 

For the data stored in das.txt, we want to test the hypothesis that these data are consistent with a long-run average of 3.

H0 :=3

H1 :3

 

Implemented directly in R using:

xbar <- mean(data$y) # sample mean

s <- sd(data$y) # sample std. deviation

n <- length(data$y) # sample size

mu <- 3 # test value

Z <- (xbar - mu)/(s/sqrt(n))

Z

[1] -24.03721

 

Now need to determine the p-value, i.e. the probability of getting a value less than -24.03721 and compare with

or the critical value to compare with the TS.

 


Have already seen how to do this using pnorm and qnorm.

2*pnorm(-24.03721)

[1] 1.135936e-127

 

 

Since p-value = 0 there is enough evidence to reject H0 and conclude that 3.

 

ALTERNATIVELY { this is a two-tailed test and  = 0:05

 

qnorm(0.975) qnorm(0.025) OR qnorm(0.975, lower.tail=FALSE)

[1] 1.959964 [1] -1.959964

 

Since TS = -24.03721 < 1:96 ) enough evidence to reject H0 and conclude that 6= 3

 

If this was a one-tailed test:

H0 :    3 H0 :    3

H1 :  > 3 H1 :  < 3

The p-value would be calculated as (note no multiplication by 2)

    >pnorm(-24.03721)

    [1] 5.679679e-128

and only one critical value is required. When there is a > sign in

H1 and

= 0:05:

    > qnorm(0.95)

    [1] 1.644854

When there is a < sign in H1 and

= 0:05:

    qnorm(0.05)

    qnorm(0.05, lower.tail=FALSE)

    [1] -1.644854

 

Confidence Intervals for the Mean

A confidence interval for the sample mean of a large sample or

when  is known is given by


 

=2 from the normal tables. Use qnorm which

gives the value of x for a speci ed value of k such that

P[X <= x] = k.

 

 


R does not have an in-built function to carry out a Z-test but does have the function t.test. A t-test is usually carried out when the sample size is small (n < 30) and the sample standard deviation is used. The test statistic is

 

 

t=X-s /n

 



and the critical values are looked up in the t-tables. However, remember that as the sample size increases, the t-distribution becomes more and more like a normal distribution. Carrying out a t-test for a large sample is equivalent to carrying out the Z-test as outlined above when the sample standard deviation is used.


Testing the previous hypothesis using the t.test function

t.test(data$y, mu=3, alternative="two.sided", conf.level=0.95)

One Sample t-test

data: data$y

t = -24.0372, df = 99, p-value < 2.2e-16

alternative hypothesis: true mean is not equal to 3

95 percent confidence interval:

2.371533 2.467379

sample estimates:

mean of x

2.419456

 

The line

One Sample t-test

is self-explanatory.

 

data: data$y

tells us what data has been used for the test.

t = -24.0372, df = 99, p-value < 2.2e-16

 

tells us the value of the test statistic (t), the degrees of freedom

(n = 1) and the p-value for the test. If the p-value < ) enough

evidence to reject H0, which is the case here.

 

alternative hypothesis: true mean is not equal to 3

tells us

 (a) that this is a two-tailed test (as we have not equal to in the alternative hypothesis) and

 (b) the value that we wanted to test whether the mean could be equal to.

95 percent confidence interval:

2.371533 2.467379

is the 95% con dence interval for these data calculated using

 




