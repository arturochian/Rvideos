The for loop 
If you have a statement or statements that you want to repeat a number of times, you can use the “for” statement to do so. 

The “for” statement loops over all elements in a list or vector: 
for (variable in sequence) statement.1 

x = numeric(100)

for( i in 1:100)
{
x[i] = i + exp(i)
}
Note the colon (:) operator generates a sequence of integers. You can use this in places other than the for loop or array indexing, too. Also note that statement.1 can be a group of statements inside braces.
The if statement 
This is used when you want to do something if a condition is true and something else otherwise. The statement looks like this: 
if ( condition ) statement 1 else statement 2 
This will execute statement 1 if the condition is true and statement 2 if the condition is false. 
if (x < 3) print("x less than 3") else print ("x not less than 3")
If you want statement 1 and / or statement 2 to consist of more than one statement, then the if construct looks like this: 
The group of statements between a { and a } are treated as one statement by the if and else. 
if (condition) { 
statement 1a 
statement 1b 
…
} else { 
statement 2a 
statement 2b 
…
} 

Other flow control statements 
R has two statements break and next: these discontinue normal execution in the middle of “for” loop. The “next” statement causes the next pass through the loop to begin at once, while “break” causes a jump to the statement immediately after the end of the loop.
Dice Roll simulation

Recall how to construct a vector of consecutive integers. Lets a construct a  vector “die” with a sequence of values from 1 to 6.

>die=1:6

Lets sample N values from this vector (using sampling with replacement). (The R function is sample(). )

This is equivalent to rolling a fair dice N times. This time let N =100.
Also, let us use the table function to analyse the outcome. 

## Initialize variables
die = 1:6
N=100

## Calculations
x=sample(die,N, replace=TRUE)
table(x)

We should get approximately equal numbers for each outcome.
Write down the mean, standard deviation and variance of your vector.
mean(x)
sd(x)
var(x)
Using control loops, we can repeat this experiment “M” times. Let us specifically study the mean of the vector, for each of the M iterations.
We will save these values in a vector “y”.

y=numeric()
M=1000;N=100
for(i in 1:M)
{
X =sample(die,N, replace =TRUE)
Xbar=mean(X)
y=c(y,Xbar)
}

What is the mean value of y? what is its standard deviation and variance?
How many values of  y are less than 3.1? how many are greater than 3.9?
Histograms
A histogram is a commonly used graphical technique consists of parallel vertical bars that graphically shows the frequency distribution of a numeric  variable. The area of each bar is equal to the frequency of items found in each class.
The command is simply hist(). 
We will revert to graphical methods in a later class.
hist(y)

Comment on the shape of this resultant histogram. (Recall the  Central Limit Theorem)




