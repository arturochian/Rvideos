
%---------------------------------------------------------------------------------%

\subsection{Vectors}
\begin{itemize}
\item $R$ operates on named data structures. The simplest such
structure is the vector, which is a single entity consisting of an
ordered collection of numbers or characters.

\item The most common types of vectors are:
\begin{itemize}
\item Numeric vectors \item Character vectors \item Logical
vectors
\end{itemize}

\item There are, of course, other types of vectors.
\begin{itemize}
\item Colour vectors - potentially useful later on.
\item Order vectors - The rankings of items in a vector.
\item Complex number vectors - not part of this course.
\end{itemize}
\end{itemize}

%---------------------------------------------------------------------------------%
\subsection{Vectors: Creating and editing a vector}
\begin{itemize}
\item From last class.
\item To create a vector, use the assignment operator ``$=$" or ( $<-$ )and
the concatenate function ``c()". \item For numeric vectors, the values
entered are simply numbers.

\begin{verbatim}
>x =c(10.4,5.6,3.1,6.4,8.9)
>
\end{verbatim}

And, from last week, we can use the ``data.entry()" function to edit our vector.
\begin{verbatim}
>data.entry(x)
>
\end{verbatim}
\end{itemize}




%---------------------------------------------------------------------------------%
\subsection{Vectors: Character \& logical vector}

\begin{itemize}
\item For character vectors, the values are simply characters,
specified with quotation marks.
\item Single quotation marks
\begin{verbatim}

>Charvec<-c(`Dog', `Cat', `Shed', `Spoon')

\end{verbatim}

\item A logical vectors is a vector whose elements are TRUE, FALSE
or NA (i.e. null)
\begin{verbatim}

>Logvec<-c(TRUE, FALSE,TRUE,TRUE )

\end{verbatim}

\end{itemize}

%---------------------------------------------------------------------------------%
\subsection{Graphical Data Entry Interface}
\begin{itemize}

\item The data.entry() command calls a spreadsheet graphical user
interface, which can be used to edit data. All changes are saved
automatically.




\item Alternatively, the edit() command calls the `R editor',
which can be used to edit specified data or the code used to
define that data.

\begin{verbatim}
x<-edit(x)
\end{verbatim}

\end{itemize}



%---------------------------------------------------------------------------------%

\subsection{Vectors: Empty vectors}

\begin{itemize}
\item Another method of creating vectors is to use the follow

\begin{itemize}
\item numeric (length = n) \item character (length = n) \item
logical (length = n)\end{itemize}

\item These commands create empty vectors, of the appropriate
kind, of length n.

\begin{verbatim}
> x<-numeric(4)
> x
[1] 0 0 0 0
\end{verbatim}

\item You can use the graphical data entry interface to populate
your data sets.
\end{itemize}


%---------------------------------------------------------------------------------%
\subsection{Vectors: Characteristics}

\begin{itemize}
\item We can use several $R$ commands to gather information about
a vector.

\begin{itemize}
\item length(x) - how many elements in a vector.  \item sum(x)-
the sum of the elements in a vector. \item prod(x) - the product
of the elements in a vector.
\end{itemize}

\item We can also find statistical information about a vector
\begin{itemize}
\item summary(x) - summary statistics of a vector.  \item mean(x)-
the mean value of a vector. \item sd(x) -  the standard deviation
of a vector.
\end{itemize}

\item Refer to the reference card for more commands to try out.
\end{itemize}

%---------------------------------------------------------------------------------%
\subsection{Vectors: Characteristics (contd)}

\begin{verbatim}
> mean(x)
[1] 6.375
> sd(y)
[1] 2.858846
> median(z)
[1] 16
> summary(x)
   Min. 1st Qu.  Median    Mean 3rd Qu.    Max.
  3.100   4.975   6.000   6.375   7.400  10.400
\end{verbatim}


%---------------------------------------------------------------------------------%
\subsection{Calculations using vectors}

\begin{itemize}
\item Calculations are performed on a vector on a case-wise basis.
The calculations are carried out on each element individually.
\begin{verbatim}
> y^2
[1]  2.56 12.25 60.84 44.89 65.61
\end{verbatim}

\end{itemize}
\begin{itemize}
\item Try the following calculations.
\begin{verbatim}
> y*z
[1]  25.6  31.5 280.8  26.8 202.5
> sum(z)
[1] 90
> sum(y^2)
[1] 186.15
> sum(y*z)
[1] 567.2
\end{verbatim}
\end{itemize}

%---------------------------------------------------------------------------------%
\subsection{Accessing vector's elements}

\begin{itemize}
\item The $n$th element of vector `x' can be accessed by
specifying its index when calling `x'.
\begin{verbatim}
>x[3]
[1] 3.1
\end{verbatim}

\item A sequence of  elements of vector `x' can be accessed by
specifying the lower and upper bound of the the range, in form
x[l:u].
\begin{verbatim}
> x[2:4]
\end{verbatim}
\end{itemize}



%---------------------------------------------------------------------------------%

\subsection{Modifying a vector}

\begin{itemize}
\item A vector can be updated by assigning an extra value to it.
\begin{verbatim}
> logvec<-c(logvec,TRUE)
> logvec
[1]  TRUE FALSE  TRUE  TRUE  TRUE
\end{verbatim}

\item A vector can be repeated $n$ times using the rep() command.
\begin{verbatim}
> rep(charvec,2)
[1] "blue" "pink" "red" "blue" "pink" "red"
\end{verbatim}

\item Omitting and deleting the $n$th element of vector `x'.
\begin{verbatim}
>charvec[-5]
>charvec <- charvec[-5]
\end{verbatim}

\end{itemize}

%---------------------------------------------------------------------------------%
\subsection{Relational operators}
A relational operator tests some kind of relation between two
entities. For $R$ the relational operators are as follows:
\begin{center}
\begin{tabular}{|c|c|c|c|}
  \hline

  Equals & == & Less or equal to  & <= \\
  \hline
  Not Equal & != & Greater than & > \\
  \hline
  Less than & < & Greater than & >= \\
  \hline
\end{tabular}
\end{center}

%---------------------------------------------------------------------------------%
\subsection{Logical operators}
\begin{itemize}
\item The logical operators are AND, OR and NOT

\item if c1 and c2 are logical expressions, then $c1 \& c2$ is
their intersection (`AND'), $c1 | c2$ is their union (`OR'), and
$!c1$ is the negation of c1.
\end{itemize}
\begin{center}
\begin{tabular}{|c|c|c|c|}
  \hline
  AND & $ \& $ & also  & $\&\&$ \\
  \hline
  OR & $|$ & also & $||$ \\
  \hline
  NOT & $!$ & &  \\
  \hline
\end{tabular}
\end{center}

%---------------------------------------------------------------------------------%
\subsection{Examples using operators}

We can use relational and logical operators to selecting elements
of a vector with specified criteria.

\begin{verbatim}
#selecting all elements of x greater than 5
>x[x>5]

#selecting all elements of x greater or equal to than 5
>x[x>=5]

#selecting all elements of x greater than 5 #or less than 3
>x[(x>5)|(x<3)]

#selecting all elements of x between 3 and 5
>x[(x>3)&(x<5)]
\end{verbatim}


%---------------------------------------------------------------------------------%
\end{document}
